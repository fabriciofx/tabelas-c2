% The MIT License (MIT)
%
% Copyright (C) 2025 Fabrício Barros Cabral
%
% Permission is hereby granted, free of charge, to any person obtaining a copy
% of this software and associated documentation files (the "Software"), to deal
% in the Software without restriction, including without limitation the rights
% to use, copy, modify, merge, publish, distribute, sublicense, and/or sell
% copies of the Software, and to permit persons to whom the Software is
% furnished to do so, subject to the following conditions:
%
% The above copyright notice and this permission notice shall be included
% in all copies or substantial portions of the Software.
%
% THE SOFTWARE IS PROVIDED "AS IS", WITHOUT WARRANTY OF ANY KIND, EXPRESS OR
% IMPLIED, INCLUDING BUT NOT LIMITED TO THE WARRANTIES OF MERCHANTABILITY,
% FITNESS FOR A PARTICULAR PURPOSE AND NON-INFRINGEMENT. IN NO EVENT SHALL THE
% AUTHORS OR COPYRIGHT HOLDERS BE LIABLE FOR ANY CLAIM, DAMAGES OR OTHER
% LIABILITY, WHETHER IN AN ACTION OF CONTRACT, TORT OR OTHERWISE, ARISING FROM,
% OUT OF OR IN CONNECTION WITH THE SOFTWARE OR THE USE OR OTHER DEALINGS IN THE
% SOFTWARE.

% Documento tipo artigo, com fonte tamanho 12
\documentclass[12pt]{article}

% Configura o tamanho do papel e das margens
\usepackage[
    a4paper,% papel A4
    left=0.5cm,% margem esquerda
    right=0.5cm,% margem direita
    top=0.5cm,% margem superior
    bottom=0.5cm, % margem inferior
    includehead,
    headheight=6mm
]{geometry}

% Cabeçalho e rodapé
\usepackage{fancyhdr}

% Caracteres / Acentos / Português do Brasil
\usepackage[utf8]{inputenc}
\usepackage[T1]{fontenc}
\usepackage[brazil]{babel}

% Símbolos matemáticos
\usepackage{amsfonts}

% Cria uma caixa de ajuste de tamanho para as tabelas
\usepackage{adjustbox}

% Tabela com mais recursos (numeração automática de linhas, cores, etc.)
\usepackage{tabularray}

% Permite o uso de cores por nome
\usepackage{xcolor}

% Começo do documento
\begin{document}

% Inclui os cabeçalhos e rodapé no documento
\pagestyle{fancy}

% Configura o cabeçalho
\fancyhead{} % Limpa todos os campos do cabeçalho
\fancyhead[R]{\textbf{Fabrício Barros Cabral}} % Configura o campo right para o texto

\section{TABELA DE DERIVADAS E INTEGRAIS}

\begin{table}[htb]
    \centering
    \begin{adjustbox}{width=0.8\textwidth}
        \begin{tblr}{
            colspec=|l|l|l|l|,
            row{1}={font=\bfseries},
            row{even}={gray9},
            cell{2-Z}{1} = {cmd=\inteval{\value{rownum}-1)}}
        }
        \hline
        No & Função & Derivada & Integral Indefinida \\ \hline
        & \textit{$f(x) = k$} & \textit{$f'(x) = 0$} & $\displaystyle\int{k\ dx} = k\int{dx} = kx + c$ \\ \hline
        & \textit{$f(x) = x$} & \textit{$f'(x) = 1$} & $\displaystyle\int{1\ dx} = 1\int{dx} = \int{dx} = x + c$ \\ \hline
        & \textit{$f(x) = ax$} & \textit{$f'(x) = a$} & $\displaystyle\int{a\ dx} = a\int{dx} = \int{dx} = ax + c$ \\ \hline
        & \textit{f(x) = $x^n$} & \textit{$f'(x) = n \cdot x^{n - 1}$} & $\displaystyle\int{x^ndx} = \frac{x^{n+1}}{n + 1} + c, n \neq -1$ \\ \hline
        & \textit{f(x) = $log_a\ x$} & \textit{$f'(x) = \displaystyle\frac{1}{x \cdot ln\ a}$} & $\displaystyle\int{\frac{1}{x \cdot ln\ a}dx} = log_a\ x + c$ \\ \hline
        & \textit{f(x) = $ln\ x$} & \textit{$f'(x) = \displaystyle\frac{1}{x}$} & $\displaystyle\int{\frac{1}{x}dx} = ln \left|x\right| + c$ \\ \hline
        & \textit{f(x) = $a^x$} & \textit{$f'(x) = a^x \cdot ln\ a$} & $\displaystyle\int{a^xdx} = \frac{a^x}{ln\ a} + c$ \\ \hline
        & \textit{$f(x) = {\rm e}^x$} & \textit{$f'(x) = {\rm e}^x$} & $\displaystyle\int{{\rm e}^xdx} = {\rm e}^x + c$ \\ \hline
        & \textit{$f(x) = sen(x)$} & \textit{$f'(x) = cos(x)$} & $\displaystyle\int{cos(x)dx} = sen(x) + c$ \\ \hline
        & \textit{$f(x) = cos(x)$} & \textit{$f'(x) = -sen(x)$} & $\displaystyle\int{sen(x)dx} = -cos(x) + c$ \\ \hline
        & \textit{$f(x) = tg(x)$} & \textit{$f'(x) = sec^2(x)$} & $\displaystyle\int{sec^2(x)dx} = tg(x) + c$ \\ \hline
        & \textit{$f(x) = cotg(x)$} & \textit{$f'(x) = -cossec^2(x)$} & $\displaystyle\int{cossec^2(x)dx} = -cotg(x) + c$ \\ \hline
        & \textit{$f(x) = sec(x)$} & \textit{$f'(x) = tg(x) \cdot sec(x)$} & $\displaystyle\int{sec(x) \cdot tg(x)dx} = sec(x) + c$ \\ \hline
        & \textit{$f(x) = cossec(x)$} & \textit{$f'(x) = -cotg(x) \cdot cossec(x)$} & $\displaystyle\int{cossec(x) \cdot cotg(x)dx} = -cossec(x) + c$ \\ \hline
        & \textit{$f(x) = arctg(x)$} & \textit{$f'(x) = \displaystyle\frac{1}{1 + x^2}$} & $\displaystyle\int{\frac{1}{1 + x^2}dx} = arctg(x) + c$ \\ \hline
        & \textit{$f(x) = arcsen(x)$} & \textit{$f'(x) = \displaystyle\frac{1}{\sqrt{1 - x^2}}$} & $\displaystyle\int{\frac{1}{\sqrt{1 - x^2}}dx} = arcsen(x) + c$ \\ \hline
        & \textit{$f(x) = arccos(x)$} & \textit{$f'(x) = \displaystyle - \frac{1}{\sqrt{1 - x^2}}$} & $\displaystyle\int{- \frac{1}{\sqrt{1 - x^2}}dx} = arccos(x) + c$ \\ \hline
        & \textit{$f(x) = ln(x + \sqrt{x^2 + 1})$} & \textit{$f'(x) = \displaystyle \frac{1}{\sqrt{1 + x^2}}$} & $\displaystyle\int{\frac{1}{\sqrt{1 + x^2}}dx} = ln\left|x + \sqrt{x^2 + 1}\right| + c$ \\ \hline
        & \textit{$f(x) = \displaystyle\frac{1}{2} \cdot ln\left|\frac{1 + x}{1 - x}\right|$} & \textit{$f'(x) = \displaystyle\frac{1}{1 - x^2}$} & $\displaystyle\int{\frac{1}{\sqrt{1 - x^2}}dx} = \displaystyle\frac{1}{2} \cdot ln\left|\frac{1 + x}{1 - x}\right| + c$ \\ \hline
        \end{tblr}
    \end{adjustbox}
\end{table}    

\section{REGRAS DE DERIVAÇÃO}

\subsection{Regra do produto}
Se $f(x) = g(x) \cdot h(x)$, então $f'(x) = g'(x) \cdot h(x) + g(x) \cdot h'(x)$

\subsection{Regra do quociente}
Se $f(x) = \displaystyle\frac{g(x)}{h(x)}$, então $f'(x) = \displaystyle\frac{g'(x) \cdot h(x) - g(x) \cdot h'(x)}{[h(x)^2]}$

\subsection{Regra da cadeia}
Se $f(x) = g[h(x)]$, então $f'(x) = g'[h(x)] \cdot h'(x)$


\section{REGRAS DE INTEGRAÇÃO}

\subsection{Integração por substituição}

Se $\displaystyle\int{f[g(x)] \cdot g'(x)dx}$, faça $u = g(x)$ e $u' = g'(x)$ e assim, teremos:
$\displaystyle\int{f[g(x)] \cdot g'(x)dx} = \displaystyle\int{f(u)du} = F(u) + c$


\subsection{Integração por partes}

$\displaystyle\int{f(x) \cdot g'(x)dx} = f(x) \cdot g(x) - \displaystyle\int{f'(x) \cdot g(x)dx}$ ou, simplesmente,
$\displaystyle\int{udv} = uv - \displaystyle\int{vdu}$\\[4mm]
\textbf{Exemplo:}\\[2mm]
$\displaystyle\int{xsen(x)dx}$\\[2mm]
Fazendo $u = x$, então $du = dx$. Fazendo $dv = sen(x)$, então $v = -cos(x)$. Assim,\\[2mm]
$\displaystyle\int{xsen(x)dx} = x \cdot -cos(x) - \left(\displaystyle\int{-cos(x)dx}\right) = -xcos(x) + \left(\displaystyle\int{cos(x)dx}\right) = -xcos(x) + sen(x) + c$

\end{document}

